\documentclass[]{article}
\usepackage{setspace}
\usepackage[margin = 1in]{geometry}
\usepackage{todonotes}
\usepackage{amssymb,amsmath}
\usepackage{ifxetex,ifluatex}
\usepackage{fixltx2e} % provides \textsubscript
\ifnum 0\ifxetex 1\fi\ifluatex 1\fi=0 % if pdftex
  \usepackage[T1]{fontenc}
  \usepackage[utf8]{inputenc}
\else % if luatex or xelatex
  \ifxetex
    \usepackage{mathspec}
  \else
    \usepackage{fontspec}
  \fi
  \defaultfontfeatures{Ligatures=TeX,Scale=MatchLowercase}
\fi
% use upquote if available, for straight quotes in verbatim environments
\IfFileExists{upquote.sty}{\usepackage{upquote}}{}
% use microtype if available
\IfFileExists{microtype.sty}{%
\usepackage{microtype}
\UseMicrotypeSet[protrusion]{basicmath} % disable protrusion for tt fonts
}{}
\usepackage{hyperref}
\hypersetup{unicode=true,
            pdftitle={Chapter 1},
            pdfauthor={Frederick Boehm},
            pdfborder={0 0 0},
            breaklinks=true}
\urlstyle{same}  % don't use monospace font for urls
\usepackage{graphicx}
\makeatletter
\def\maxwidth{\ifdim\Gin@nat@width>\linewidth\linewidth\else\Gin@nat@width\fi}
\def\maxheight{\ifdim\Gin@nat@height>\textheight\textheight\else\Gin@nat@height\fi}
\makeatother
% Scale images if necessary, so that they will not overflow the page
% margins by default, and it is still possible to overwrite the defaults
% using explicit options in \includegraphics[width, height, ...]{}
\setkeys{Gin}{width=\maxwidth,height=\maxheight,keepaspectratio}
\IfFileExists{parskip.sty}{%
\usepackage{parskip}
}{% else
\setlength{\parindent}{0pt}
\setlength{\parskip}{6pt plus 2pt minus 1pt}
}
\setlength{\emergencystretch}{3em}  % prevent overfull lines
\providecommand{\tightlist}{%
  \setlength{\itemsep}{0pt}\setlength{\parskip}{0pt}}
\setcounter{secnumdepth}{0}
% Redefines (sub)paragraphs to behave more like sections
\ifx\paragraph\undefined\else
\let\oldparagraph\paragraph
\renewcommand{\paragraph}[1]{\oldparagraph{#1}\mbox{}}
\fi
\ifx\subparagraph\undefined\else
\let\oldsubparagraph\subparagraph
\renewcommand{\subparagraph}[1]{\oldsubparagraph{#1}\mbox{}}
\fi

%%% Use protect on footnotes to avoid problems with footnotes in titles
\let\rmarkdownfootnote\footnote%
\def\footnote{\protect\rmarkdownfootnote}

%%% Change title format to be more compact
\usepackage{titling}
\usepackage[
    backend=biber,
    style=authoryear,
    natbib=true,
    url=true, 
    doi=true,
    eprint=true
]{biblatex}
\addbibresource{ch1.bib}
\addbibresource{research.bib}

% Create subtitle command for use in maketitle
\newcommand{\subtitle}[1]{
  \posttitle{
    \begin{center}\large#1\end{center}
    }
}

\setlength{\droptitle}{-2em}

  \title{Chapter 1}
    \pretitle{\vspace{\droptitle}\centering\huge}
  \posttitle{\par}
    \author{Frederick Boehm}
    \preauthor{\centering\large\emph}
  \postauthor{\par}
      \predate{\centering\large\emph}
  \postdate{\par}
    \date{12/31/2018}


\begin{document}
\doublespacing
\maketitle

\listoftodos
\listoffigures
\listoftables


\begin{enumerate}
\def\labelenumi{\arabic{enumi}.}
\tightlist
\item
  Complex traits \& QTL mapping
\end{enumerate}

\begin{itemize}
\tightlist
\item
  look at Karl \& Saunak's chapter 1
\item
  goal is to motivate study of complex traits with QTL mapping
\end{itemize}

Identification of genes that affect measurable phenotypes has a long and
successful history in model organisms. Complex traits include classical
clinical phenotypes such as systolic blood pressure and body weight as
well as newly measurable biomolecular phenotypes like gene expression
levels, protein concentrations, and lipid levels. Understanding the
genetic underpinnings of such traits may inform many areas of biology,
medicine, and public health.

The first reported QTL study is from 1923, 30 years before the discovery of the structure of DNA \citep{watson1953molecular}. \citet{sax1923association} examined seed weight for the common bean (\emph{Phaseolus vulgaris}) in an F\(_2\) intercross. He assigned each F\(_2\) subject to a gene class by examining its seed color patterns.

\citet{lander1989mapping} kickstarted modern QTL mapping methods research with their seminal report in the late 1980s. Their article

Goals of a QTL study depend on its scientific context. Often a
researcher seeks to identify genomewide positions of QTL for each trait
of interest. In some studies, the total number of QTL for a trait may be
more interesting than the QTL positions.

A QTL study begins with a scientific question and the choice of a study
design. Elements to consider include the mating design, phenotyping
plan, genotyping plan, and statistical analysis methods. For most of the
last century, attaining clinical phenotypes, such as body weight, was
much less costly than genotyping. This setting sometimes led researchers
to selectively genotype only those organisms with extreme phenotype
values. With diminishing costs for both genotyping and phenotyping, many
recent studies genotype and phenotype all subjects.

Since the 1980s, researchers have written and shared computer software
for QTL studies. Early efforts included MAPMAKER, QTL Cartographer,

Since the early 2000s, the ``qtl'' R package has been a
state-of-the-art resource for QTL mapping studies. It is open-source,
free to download, well documented, and well supported.

\textbf{Mamm Genome. 1999 Apr;10(4):327-34. Overview of QTL mapping
software and introduction to map manager QT. Manly KF1, Olson JM.}

\begin{enumerate}
\def\labelenumi{\arabic{enumi}.}
\setcounter{enumi}{1}
\tightlist
\item
  Univariate QTL scan in two-parent crosses
\end{enumerate}

\begin{itemize}
\tightlist
\item
  goal: explain a QTL ``scan'' in backcross \& intercross. Use
  pictures/diagrams of chromosome(s) What are the genotypes of
  individuals from a backcross? What are the genotypes of individuals
  from an intercross (F2)?
\end{itemize}

One widely used mating design is the backcross. Although variations are possible, one typically begins with two inbred lines. Let's designate the two lines as "A" and "B". Mating of lines A and B leads to offspring, which we designate as F$_1$ (to denote the first filial generation). The F$_1$ offspring, then, mate with the A line (ie, the parental line) to produce N$_2$ subjects, where the letter "N" denotes the offspring from a backcross and subscript 2 reflects the generation number.

We assume that we're working with diploid organisms, so that every subject has two copies - which need not be identical - of each chromosome. Let's assume, further, that we're working with mice, so that every organism has 20 chromosome pairs. 

Inbred lines, by definition, are homozygous at all markers. Let's designate, for a given marker, the A line to have two copies of allele A and the B line to have two copies of allele B. 

Let's consider the genetic makeup of the F$_1$ subjects. We first must examine gametogenesis, the process of producing gametes, or reproductive cells, in the parents. Gametogenesis results in production of haploid cells, \emph{i.e.}, cells with only one copy of every chromosome. Two gametes, one from each parent, unite to form a diploid zygote. All F$_1$ subjects are genetically identical. For every chromosome pair, they inherited one copy of the A chromosome (from the A parent) and one copy of the B chromosome (from the B parent). In other words, every F$_1$ subject has genotype AB at every marker, where A the allele from the A parent and B the allele from the B parent. The N$_2$ generation, then, has subjects with either AA or AB genotypes at a given marker. 

To understand whether a given subject has AA or AB genotype at a given marker, we need to consider the fact that gametogenesis involves crossover events before the diploid cells divide into haploid cells. A crossover event results in a swapping of DNA between the two copies of a chromosome. In inbred lines, this swapping of DNA between the two copies of a chromosome is undetectable with marker genotyping, because both chromosomes have the same allele (either two As or two Bs in our example). However, the F$_1$ subjects, with AB marker genotypes, have distinct alleles at every marker. Thus, marker genotyping has the potential to detect crossover events that occur during gametogenesis in the F$_1$ by examining marker genotypes in the N$_2$ subjects. A picture helps to clarify this idea (Figure \ref{}).

\todo{Add figure here AND add text explaining the figure here}

Another widely used two-parent cross is the "intercross". In it, two inbred lines again mate to produce F$_1$ subjects. Then, however, two F$_1$ subjects mate to produce F$_2$ subjects. The crossover events that occur during gametogenesis in the F$_1$ subjects gives rise to the three genotypes observed in F$_2$ subjects: AA, AB, and BB. We present a diagram for the intercross in Figure \ref{}.

Both 







Lander \& Botstein 1989 Haley \& Knott 1992 Martinez \& Curnow 1992

Soller et al 1976

\begin{enumerate}
\def\labelenumi{\arabic{enumi}.}
\setcounter{enumi}{2}
\tightlist
\item
  Multivariate QTL scan in two-parent crosses
\end{enumerate}

Jiang \& Zeng 1995 Knott \& Haley 2000

Jiang and Zeng developed multivariate methods for QTL mapping in
two-parent crosses. They devised a multivariate analog of Zeng's
composite interval mapping (Zeng 1993, 1994). This strategy treats
phenotypes as arising from a mixture of normal distributions in which
each genotype class has distinct distribution parameters. Within this
multivariate mapping framework, Jiang and Zeng developed the first test
of pleiotropy vs.~separate QTL.

Jiang and Zeng first developed a test for pleiotropy vs.~separate QTL in
two-parent crosses. They developed it in the context of their work in
multivariate QTL studies with composite interval mapping. They framed
the scientific question of whether two traits are affected by a single,
shared locus or by two distinct loci in terms of two statistical
hypotheses. The null hypothesis states that there is a single
pleiotropic locus that affects both traits, while the alternative
hypothesis is that there are two distinct but nearby loci, with each
locus affecting exactly one trait.

Knott and Haley subsequently reported methods for testing pleiotropy
vs.~separate QTL in two-parent crosses. Knott and Haley integrated their
earlier work on univariate QTL mapping with marker regression with Jiang
and Zeng's multivariate methods to develop a test of pleiotropy
vs.~separate QTL with multivariate marker regression methods.

Jiang and Zeng (1995)

One disadvantage of multivariate compositive interval mapping is the
computing requirements for the iterative expectation-maximization
procedures needed for parameter estimation. This prompted Knott and
Haley (2000) to develop a marker regression-based approximation to
multivariate composite interval mapping. Knott and Haley (2000) used a
multivariate linear model for simultaneous mapping of multiple traits.
They also presented a multivariate marker regression-based test of
pleiotropy vs. separate QTL. This test is suitable for subjects that are
equally related to each other, like the collection of offspring in an
F\(_2\) intercross of two inbred lines.

\begin{enumerate}
\def\labelenumi{\arabic{enumi}.}
\setcounter{enumi}{3}
\tightlist
\item
  Multiparental populations
\end{enumerate}

\begin{itemize}
\tightlist
\item
  what are they? Breeding design for CC \& DO. Why use them?
\end{itemize}

While QTL mapping studies in the 1990s contributed to many advances in
genetics and biology, complex trait researchers recognized the mapping
resolution limitations in crosses involving two inbred lines. Seeking
greater precision for QTL positions, scientific communities collectively
decided to pool their expertise and resources into community-supported
and community-maintained model organism mapping populations. These new
mapping populations would incorporate genetic material from more than
two inbred founder lines. The accumulated meiotic recombination events
over many generations would enhance mapping resolution over previously
available populations.

Products of these community-based efforts include the Collaborative
Cross and Diversity Outbred populations from mouse researchers and
Drosophila Synthetic Population Resource \citep{king2012genetic} and  in the fruit fly
scientific community. In subsequent years, scientists created
multiparental populations in many other organisms, including tomato, rice, maize \citep{lehermeier2014usefulness}, wheat \citep{mackay2014eight, huang2012multiparent, milner2016multiparental}, Arabidopsis \citep{}, apple \citep{allard2016detecting}

\begin{enumerate}
\def\labelenumi{\arabic{enumi}.}
\setcounter{enumi}{4}
\tightlist
\item
  Univariate QTL mapping in MPP
\end{enumerate}

\begin{itemize}
\tightlist
\item
  contrast with univariate QTL mapping in two-parent crosses
\end{itemize}

\begin{enumerate}
\def\labelenumi{\arabic{enumi}.}
\setcounter{enumi}{5}
\tightlist
\item
  Multivariate mapping in mpp
\end{enumerate}

6A.

\begin{enumerate}
\def\labelenumi{\arabic{enumi}.}
\setcounter{enumi}{6}
\tightlist
\item
  Testing pleiotropy vs separate QTL in MPP
\end{enumerate}

Testing pleiotropy vs.~separate

7A. allele effects plots to discern pleiotropy v separate QTL

King et al 2012 Macdonald \& Long 2007 \emph{maybe do a citation search
on these 2 articles to see who has used their ideas} CAPE software
package - what exactly is the CAPE method???

\begin{enumerate}
\def\labelenumi{\arabic{enumi}.}
\setcounter{enumi}{7}
\tightlist
\item
  Testing pleiotropy vs separate QTL to dissect an expression trait
  hotspot Tian et al.~2016. ?Schadt et al.~2005?
\end{enumerate}


\end{document}
